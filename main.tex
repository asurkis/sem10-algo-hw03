\documentclass{article}

\usepackage[T2A]{fontenc}
\usepackage[utf8]{inputenc}
\usepackage[russian]{babel}
\usepackage{fullpage}
\usepackage{indentfirst}

\usepackage{amsmath}
\usepackage{amsfonts}
\usepackage{amsthm}
\usepackage{float}
\usepackage{tikz}

\usepackage[outputdir=out]{minted}
%\usepackage{algorithmicx}
%\usepackage{algpseudocodex}
%\usepackage{algorithm2e}
\usepackage{hyperref}
\usepackage{enumitem}
\usepackage{array}

%\setlength{\parindent}{1.25cm}
%\renewcommand{\baselinestretch}{1.5}
\setlength{\parskip}{6pt}

\begin{document}
    \theoremstyle{definition}
    \newtheorem*{definition}{Определение}
    \newtheorem{theorem}{Теорема}
    \newtheorem{statement}{Утверждение}
    \newtheorem{lemma}{Лемма}

    \newcommand{\N}{\mathbb{N}}
    \newcommand{\R}{\mathbb{R}_{>0}}
    \renewcommand{\O}{\mathcal{O}}
    \renewcommand{\o}{o}
    \newcommand{\Const}{\mathit{Const}}
    \newcommand{\Mod}{~\text{mod}~}
    \newcommand{\thus}{\Rightarrow}

    \newcommand{\paren}[1]{\left ( #1 \right )}
    \newcommand{\brackets}[1]{\left [ #1 \right ]}
    \newcommand{\braces}[1]{\left \{ #1 \right \}}
    \newcommand{\floor}[1]{\left \lfloor #1 \right \rfloor}
    \newcommand{\ceil}[1]{\left \lceil #1 \right \rceil}
    \newcommand{\abs}[1]{\left | #1 \right |}

    \hfill
    \begin{tabular}{ll}
        Студент: & Антон Суркис \\
        Группа:  & M4141        \\
        Дата:    & \today       \\
    \end{tabular}
    \hrule

    %% \section{Задача 1}
%   \item \onlygroup{Кравченко и Крыштаповича}\\
%   Дана пустая целочисленная плоскость. Нужно online за $\O(\log n)$ выполнять запросы:
%   \begin{itemize}
%       \item Добавить/удалить точку $(x_i, y_i)$
%       \item Вывести любую точку из области $l_i < x < r_i,~ y < t_i$
%   \end{itemize}
 \clearpage % Кравченко и Крыштапович
    \section{Задача 2}
Дан массив из $n$ чисел. Все операции по модулю фиксированного $M$.
Online запросы:
\begin{itemize}
    \item посчитать произведение всех чисел на отрезке $[L, R]$;
    \item присвоить значение $x$ всем числам на отрезке $[L, R]$.
\end{itemize}

Убедитесь, что ваше решение работает именно за $\O(\log n)$.

\subsection{Решение}
Рассмотрим следующий код на Rust.
В нём используются две ключевые идеи:
\begin{itemize}
    \item Если добить длину массива до $2^{\ceil{\log_2 n}}$, то все отрезки дерева отрезков будут степенями двойки,
    при этом $2^{\ceil{\log_2 n}} \in \Theta(n)$.
    \item Можно предподсчитать все $x^{2^k} \Mod M$ за $\O(\log n)$,
    поместить их в список в порядке убывания степени,
    и при спуске по дереву передавать хвост списка в вершине.
    Сам список при этом не копируется, а вершины хранятся со счётчиком ссылок,
    поэтому такая операция занимает $\O(1)$ времени на каждый уровень.
\end{itemize}

\inputminted[firstline=5]{rust}{code/src/task02/mod.rs}

Искомый алгоритм найден.
 \clearpage
    %% \section{Задача 3}
%   \item \onlygroup{Кравченко и Крыштаповича}\\
%       Дан массив из $n$ элементов, можно сделать предобработку за $\O(n \log n)$. Нужно online за $\O(\log n)$ обрабатывать запросы: количество
%     различных чисел на отрезке $[L, R]$.\\
%     \emph{Подсказка: постройте массив $prev[i] = \max \Big\lbrace \{ j ~|~ j < i ~\land~ a[j] = a[i] \} \cup \{-1\} \Big\rbrace$}
 \clearpage % Кравченко и Крыштапович
    \section{Задача 4}
Даны число $K$ и изначально пустая последовательность. Вам поступает $n$ запросов, каждый одного из двух типов:
\begin{itemize}
    \item \texttt{append(x)} --- дописать элемент $x$ в конец последовательности
    \item \texttt{rev()} ---
    развернуть $K$ последних элементов последовательности (если в данный момент их всего меньше $K$, то
    развернуть всю последовательность).
\end{itemize}

Вам нужно один раз после всех запросов вывести получившуюся последовательность.
\begin{enumerate}[label=(\alph*)]
    \item $\O(n \log{n})$
    \item $\O(n)$
\end{enumerate}

\subsection{Решение}
\subsubsection{Пункт~(a)}
Будем хранить нашу последовательность в виде АВЛ-дерева.
В каждой вершине $v$ дополнительно будем хранить информацию $r(v)$, нужно ли развернуть поддерево этой вершины целиком.
$P(v)$ --- левый ребёнок, $S(v)$ --- правый ребёнок.
Назовём нормализацией вершины следующие действия:
\begin{algorithmic}
    \If{$r(v)$}
        \State $r(P(v)) \gets \lnot r(P(v))$
        \State $r(S(v)) \gets \lnot r(S(v))$
        \State \Call{swap}{$P(v), S(v)$}
        \State $r(v) \gets 0$
    \EndIf
    \LComment{$r(v) = 0 \quad \O(1)$, баланс дерева не нарушается}
\end{algorithmic}

Будем проводить нормализацию при \emph{каждом} спуске в вершину, до перехода к её детям.

Тогда вставка \texttt{append(x)} --- это обычная вставка (с учётом нормализации)
самого правого элемента $x$ в АВЛ-дерево.
$\O(\log L)$ на запрос, где $L$ --- текущая длина последовательности.

Разворачивание \texttt{rev()}:
\begin{algorithmic}
    \LComment{Применяем нормализацию также и в \textsc{split} и \textsc{merge}}
    \State $T_1, T_2 \gets$ \Call{split}{корень, индекс с конца не больше $K$}
    \Comment{$\O(\log L)$}
    \LComment{
        Поскольку до этого мы нормализовывали вершину $T_2$,
        то эквивалентно $r(T_2) \gets 1$}
    \State $r(T_2) \gets \lnot r(T_2)$
    \Comment{$\O(1)$}
    \State $T \gets$ \Call{merge}{$T_1, T_2$}
    \Comment{$\O(\log L)$}
\end{algorithmic}

Тогда разворачивание --- тоже $\O(\log L)$ на запрос.

Вывод итоговой последовательности --- проход по дереву с нормализацией.
Он производится за $\O(L)$.

Тогда в результате весь алгоритм требует $n \cdot \O(\log L) + \O(L) \subset \O(n \log n)$
времени.

Искомый алгоритм найден.

Также решение из задачи~7 --- более общий вариант этого решения.

\subsubsection{Пункт~(b)}
Заведём вектор ответа и двустороннюю очередь для запросов.
Очевидно, что развернуть двустороннюю очередь можно просто решив добавлять и удалять
элементы из противоположного конца, т.е. за $\O(1)$.
Тогда в очереди храним последние $K$ элементов, а в векторе --- уже неизменяемые $n - K$.

Этому соответствует следующий код на Rust:
\inputminted[firstline=5]{rust}{code/src/task04/mod.rs}

Очевидно, этот алгоритм работает за $\O(n)$ и получает нужную последовательность.

Искомый алгоритм найден.
 \clearpage
    \section{Задача 5}
Дан массив из $n$ \textbf{различных} целых чисел.
Online запросы вида <<вывести $k$-ю порядковую статистику на отрезке от $L$ до $R$>>.
\begin{enumerate}[label=(\alph*)]
    \item $\langle \O(n \log{n}), \O(\log^2 n) \rangle$.
    \item $\langle \O(n \log{n}), \O(\log n) \rangle$.
\end{enumerate}
 \clearpage
    \section{Задача 6}
% \onlygroup{Мишунина}
Дан набор из $n$ троек целых чисел $(x_i, y_i, z_i)$.
Надо уметь находить online за $\langle \O(n \log{n}), \O(\log n) \rangle$ величину:
$$\min \{ z_i \mid L \le x_i \le R \land B \le y_i \le U \}$$
 \clearpage
    \section{Задача 7 ,,\texttt{C-c-c-combo!}``}

Напишите на \texttt{Haskell} реализацию персистентной структуры,
умеющей отвечать на запросы:
\begin{itemize}
    \item \texttt{insert(i, x)}, $0 \le i \le L$ ($L$ --- текущая длина) --- вставить $x$ в позицию $i$. Все элементы в $i$ и правее сдвигаются на $1$ вправо.
    \item \texttt{delete(i)} --- удалить элемент из позиции $i$. Все правее сдвигается на $1$ влево.
    \item \texttt{add(l, r, value)} --- добавить $value$ ко всем $x$, для которых $l \le i \le r$
    \item \texttt{set(l, r, value)} --- установить в $value$ все $x$, для которых $l \le i \le r$
    \item \texttt{sum(l, r)} --- сумма всех $x$, для которых $l \le i \le r$
    \item \texttt{reverse(l, r)} --- изменить порядок всех $x$, для которых $l \le i \le r$, на обратный
\end{itemize}

В начальный момент структура пустая. Время работы на запрос $\O(\log{L})$, online.

\subsection{Решение}
Код также доступен по ссылке: \\
\url{https://github.com/asurkis/sem10-algo-hw03/blob/main/hs/src/Lib.hs}

\inputminted{haskell}{hs/src/Lib.hs}
 \clearpage
\end{document}
